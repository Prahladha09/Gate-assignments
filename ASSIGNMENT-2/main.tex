
\let\negmedspace\undefined
\let\negthickspace\undefined
\documentclass[journal]{IEEEtran}
\usepackage[a5paper, margin=10mm, onecolumn]{geometry}
%\usepackage{lmodern} % Ensure lmodern is loaded for pdflatex
\usepackage{tfrupee} % Include tfrupee package

\setlength{\headheight}{1cm} % Set the height of the header box
\setlength{\headsep}{0mm}     % Set the distance between the header box and the top of the text

\usepackage{gvv-book}
\usepackage{gvv}
\usepackage{cite}
\usepackage{amsmath,amssymb,amsfonts,amsthm}
\usepackage{algorithmic}
\usepackage{graphicx}
\usepackage{textcomp}
\usepackage{xcolor}
\usepackage{txfonts}
\usepackage{listings}
\usepackage{enumitem}
\usepackage{mathtools}
\usepackage{gensymb}
\usepackage{comment}
\usepackage[breaklinks=true]{hyperref}
\usepackage{tkz-euclide} 
\usepackage{listings}
\usepackage{gvv}                                        
\def\inputGnumericTable{}                                 
\usepackage[latin1]{inputenc}                                
\usepackage{color}                                            
\usepackage{array}  
\usepackage{longtable}                                       
\usepackage{calc}                                             
\usepackage{multirow}                                         
\usepackage{hhline}                                           
\usepackage{ifthen}                                           
\usepackage{lscape}
\usepackage{multicol}
\begin{document}

\bibliographystyle{IEEEtran}
\vspace{3cm}

\title{GATE-2009-PH}
\author{AI24BTECH11024-Pappuri Prahladha}
% \bigskip
{\let\newpage\relax\maketitle}

\renewcommand{\thefigure}{\theenumi}
\renewcommand{\thetable}{\theenumi}
\setlength{\intextsep}{10pt} % Space between text and floats


\numberwithin{equation}{enumi}
\numberwithin{figure}{enumi}
\renewcommand{\thetable}{\theenumi}
\begin{enumerate}[start=37]
\item \hspace{1mm}
\begin{figure}[!ht]
\centering
\resizebox{0.50\textwidth}{!}{% % Adjust the scale here
\begin{circuitikz}
\tikzstyle{every node}=[font=\Large] % Adjust font size if needed

\draw (5.5,11) to[battery] (5.5,7.75);

\draw (8.25,11) to[R] (8.25,7.75);
\draw (5.5,11) to[R] (8.25,11);
\draw (8.25,11) to[R] (11.75,11);
\draw (11.75,11) to[R] (11.75,7.75);
\draw (5.5,7.75) to[short] (11.75,7.75);

\node [font=\large] at (10,11.5) {10$\Omega$};
\node [font=\large] at (7,11.5) {10$\Omega$};
\node [font=\large] at (4.75,9.25) {15V};
\node [font=\large] at (12.25,9.25) {$R_{L}$};
\node [font=\large] at (9,9.25) {10$\Omega$};

\end{circuitikz}
}
\end{figure}

Assuming an ideal voltage source. Thevenin's resistance and Thevenin's voltage respectively for the above circuit are 
\begin{multicols}{4}
\begin{enumerate}
    \item $15\ohm$ and 7.5\textbf{V}
    \item $20\ohm$ and 5\textbf{V}
    \item $10\ohm$ and 10\textbf{V}
    \item $30\ohm$ and 15\textbf{V}
\end{enumerate}
\end{multicols}
\item Let $ |n\rangle $ and $ |p\rangle $ denote the isospin states with $ I = \frac{1}{2} $, $ I_3 = \frac{1}{2} $ and $ I = \frac{1}{2} $, $ I_3 = -\frac{1}{2} $ of a nucleon respectively. Which one of the following two-nucleon states has $ I = 0, I_3 = 0 $?
\begin{multicols}{2}
\begin{enumerate}
    \item $ \frac{1}{\sqrt{2}} \left( | nn \rangle - | pp \rangle \right) $
    \item $ \frac{1}{\sqrt{2}} \left( | nn \rangle + | pp \rangle \right) $
    \item $ \frac{1}{\sqrt{2}} \left( | np \rangle - | pn \rangle \right) $
    \item $ \frac{1}{\sqrt{2}} \left( | np \rangle + | pn \rangle \right) $
\end{enumerate}
\end{multicols}
\item An amplifier of gain 1000 is made into a feedback amplifier by feeding $9.9\%$ of its output voltage in series with the input opposing. If $f_{L}=20Hz$ and $f_{H}=200kHz$ for the amplifier without feedback, then due to the feedback
\begin{enumerate}
    \item the gain decreases by 10 times
    \item the output resistance increases by 10 times
    \item the $f_{H}$ increases by 100 times
    \item the input resistance decreases by 100 times
\end{enumerate}
\item \hspace{1cm}

\begin{figure}[H]
\centering
\resizebox{0.50\textwidth}{!}{% % Adjust the scale here
\begin{circuitikz}
\tikzstyle{every node}=[font=\Large] % Adjust font size if needed

\node [font=\LARGE] at (10,8) {};
\node [font=\Large] at (10,7) {};
\node [font=\large] at (10,8) {\textbf{}};

\draw (3,10) to[battery] (3,3.75);
\draw (11.5,10) to[R] (11.5,3.75);
\draw (3,10) to[R] (7.25,10);
\draw (3,3.75) to[short] (11.5,3.75);
\draw (7.25,10) to[short] (11.5,10);
\draw [line width=0.8pt, ->, >=Stealth] (2.5,6.5) -- (3.75,7.25);
\draw [short] (7.25,5) -- (7,5.25);
\draw [short] (7.25,5) -- (7.5,5.25);
\draw [short] (10.75,10) -- (10.5,10.25);
\draw [short] (10.75,10) -- (10.5,9.75);
\draw [short] (3.5,10) -- (3.25,10.25);
\draw [short] (3.5,10) -- (3.25,9.75);
\draw (7.25,3.75) to[empty Zener diode] (7.25,10);

\node [font=\normalsize] at (2,7.25) {$V_{in}$};
\node [font=\normalsize] at (2,6.75) {$15-25 V $};
\node [font=\Large] at (3.25,10.5) {I};
\node [font=\normalsize] at (5.25,10.5) {$R_s = 1 \, k\Omega$};
\node [font=\large] at (10.5,10.5) {$I_L$};
\node [font=\normalsize] at (12.25,7) {$R_L$};
\node [font=\normalsize] at (8,5.25) {$I_Z $};
\node [font=\normalsize] at (8.25,6.75) {$V_Z = 10 \, V$};

\end{circuitikz}
}%
\end{figure}



Pick the correct statement based on the above circuit.
\begin{enumerate}
    \item The maximum Zener current $I_{Z\brak{max}}$, when $R_{L}=10k\ohm$ is 15mA
    \item The minimum Zener current $I_{Z\brak{min}}$, when $R_{L}=10k\ohm$ is 5mA
    \item With $V_{in}=20V$,$I_{L}=I_{Z}$ when $R_{L}=2k\ohm$
    \item The power dissipated across the Zener when $R_{L}=10k\ohm$ and $V_{in}=20V$ is 100mW
\end{enumerate}
\item The disintegration energy is defined to be the difference in the rest energy between the initial and final states.Consider the following process:\\$^{240}_{94}Pu\rightarrow ^{236}_{92}U+^{4}_{2}He$.\\The emitted $\alpha$ particle has a kinetic energy 5.17 MeV.The value of disintegration energy is
\begin{multicols}{4}
\begin{enumerate}
    \item 5.26 MeV
    \item 5.17 MeV
    \item 5.08 MeV
    \item 2.59 MeV
\end{enumerate}
\end{multicols}
\item A classical particle is moving in an external potential field $V\brak{x,y,z}$ which is invariant under the following infinitesimal transformations
\begin{align*}
x &\rightarrow x^{\prime} = x + \delta x,\\
y &\rightarrow y^{\prime} = y + \delta y,\\
\begin{pmatrix} x \\ y \end{pmatrix} &\rightarrow \begin{pmatrix} x^{\prime} \\ y^{\prime} \end{pmatrix}=R_{z}\begin{pmatrix} x \\ y \end{pmatrix},
\end{align*}
where $R_z$ is the matrix corresponding to rotation about the z axis. The conserved quantities are (the symbols have their usual meaning)
\begin{multicols}{4}
\begin{enumerate}
\item $P_{x}, P_{z}, L_{z}$
\item $P_{x}, P_{y}, L_{z}, E$
\item $P_{y}, L_{z}, E$
\item $P_{y}, P_{z}, L_{x}, E$
\end{enumerate} 
\end{multicols}
\item The spin function of a free particle, in the basis in which $ S_{z} $ is diagonal, can be written as $ \begin{pmatrix} 1 \\ 0 \end{pmatrix} $ and $ \begin{pmatrix} 0 \\ 1 \end{pmatrix} $ with eigenvalues $ +\frac{\hbar}{2} $ and $ -\frac{\hbar}{2} $, respectively. In the given basis, the normalized eigenfunction of $ S_{y} $ with eigenvalue $ -\frac{\hbar}{2} $ is:
\begin{multicols}{4}
\begin{enumerate}
\item $  \frac{1}{\sqrt{2}}\begin{pmatrix} 1 \\ i \end{pmatrix}$ 
\item $  \frac{1}{\sqrt{2}}\begin{pmatrix} 0 \\ i \end{pmatrix}$ 
\item $  \frac{1}{\sqrt{2}}\begin{pmatrix} i \\ 0 \end{pmatrix} $
\item $  \frac{1}{\sqrt{2}}\begin{pmatrix} i \\ 1 \end{pmatrix}$
\end{enumerate}
\end{multicols}

\item  Let $ \hat{A} $ and $ \hat{B} $ represent two physical characteristics of a quantum system. If $ \hat{A} $ is Hermitian, then for the product $\hat{A}\hat{B} $ to be Hermitian, it is sufficient that:
\begin{enumerate}
\item   $\hat{B}$ is Hermitian
\item   $\hat{B}$ is anti-Hermitian
\item   $\hat{B}$  Hermitian and  $\hat{A}$ and  $\hat{B}$  commute
\item   $\hat{B}$  is Hermitian and  $\hat{A}$  and $\hat{B}$ anti-commute
\end{enumerate}
\item Consider the set of vectors in three-dimensional real vector space\\
$\mathbb{R}^{3},S=\cbrak{\brak{1,1,1},\brak{1,-1,1},\brak{1,1,-1}}$. Which one of the following statements is true?
\begin{enumerate}
    \item $S$ is not a linearly independent set.
    \item $S$ is a basis for $\mathbb{R}^{3}$.
    \item The vectors in $S$ are orthogonal.
    \item An orthogonal set of vectors cannot be generated from $S$.
\end{enumerate}
\item For a Fermi gas of $N$ particles in three dimensions at $T=0k$,the Fermi energy,$E_{F}$ is proportional to
\begin{enumerate}
    \item $N_{\frac{2}{3}}$
    \item $N_{\frac{3}{2}}$
    \item $N_{3}$
    \item $N_{3}$
\end{enumerate}
\item  The Lagrangian of a diatomic molecule is given by $L = \frac{m}{2}\brak{\dot{x_1}^{2} + \dot{x_2}^{2}} - \frac{k}{2}x_{1}x_{2},$
where $m$ is the mass of each of the atoms and $x_{1}$ and $x_{2}$ are the displacements of the atoms measured from the equilibrium position, and $k > 0$. The normal frequencies are 
\begin{multicols}{4}
\begin{enumerate}
\item $\pm \left(\frac{k}{m}\right)^{1/2}$ 
\item $\pm \left(\frac{k}{m}\right)^{1/4}$ 
\item $\pm \left(\frac{k}{2m}\right)^{1/4}$ 
\item $\pm \left(\frac{k}{2m}\right)^{1/2}$
\end{enumerate}
\end{multicols}

\item  A particle is in the normalized state $|\psi\rangle$, which is a superposition of the energy eigenstates $|E_{0} = 10eV\rangle$ and $|E_{1} = 30eV\rangle$. The average value of energy of the particle in the state $|\psi\rangle$is 20 eV. The state $|\psi\rangle$ is given by 
\begin{multicols}{2}  
\begin{enumerate}
\item $\frac{1}{2} |E_{0} = 10eV\rangle + \frac{\sqrt{3}}{4} |E_{1} = 30eV\rangle $
\item $\frac{1}{\sqrt{3}}|E_{0} = 10eV\rangle + \frac{\sqrt{2}}{3} |E_{1} = 30eV\rangle$ 
\item $\frac{1}{2} |E_{0} = 10 eV\rangle - \frac{\sqrt{3}}{4}|E_{1} = 30 eV\rangle$ 
\item $\frac{1}{\sqrt{2}}|E_{0} = 10eV\rangle - \frac{1}{\sqrt{2}} |E_{1} = 30eV\rangle$
\end{enumerate}
\end{multicols}













\end{enumerate}


\end{document}
