
\let\negmedspace\undefined
\let\negthickspace\undefined
\documentclass[journal]{IEEEtran}
\usepackage[a5paper, margin=10mm, onecolumn]{geometry}
%\usepackage{lmodern} % Ensure lmodern is loaded for pdflatex
\usepackage{tfrupee} % Include tfrupee package

\setlength{\headheight}{1cm} % Set the height of the header box
\setlength{\headsep}{0mm}     % Set the distance between the header box and the top of the text

\usepackage{gvv-book}
\usepackage{gvv}
\usepackage{cite}
\usepackage{amsmath,amssymb,amsfonts,amsthm}
\usepackage{algorithmic}
\usepackage{graphicx}
\usepackage{textcomp}
\usepackage{xcolor}
\usepackage{txfonts}
\usepackage{listings}
\usepackage{enumitem}
\usepackage{mathtools}
\usepackage{gensymb}
\usepackage{comment}
\usepackage[breaklinks=true]{hyperref}
\usepackage{tkz-euclide} 
\usepackage{listings}
\usepackage{gvv}                                        
\def\inputGnumericTable{}                                 
\usepackage[latin1]{inputenc}                                
\usepackage{color}                                            
\usepackage{array}  
\usepackage{longtable}                                       
\usepackage{calc}                                             
\usepackage{multirow}                                         
\usepackage{hhline}                                           
\usepackage{ifthen}                                           
\usepackage{lscape}
\usepackage{multicol}
\begin{document}

\bibliographystyle{IEEEtran}
\vspace{3cm}

\title{GATE-2011-AE}
\author{AI24BTECH11024-Pappuri Prahladha}
% \bigskip
{\let\newpage\relax\maketitle}

\renewcommand{\thefigure}{\theenumi}
\renewcommand{\thetable}{\theenumi}
\setlength{\intextsep}{10pt} % Space between text and floats


\numberwithin{equation}{enumi}
\numberwithin{figure}{enumi}
\renewcommand{\thetable}{\theenumi}
\begin{enumerate}
\item  The possible set of eigenvalues of a $ 4 \times 4 $ skew-symmetric orthogonal real matrix is
\begin{multicols}{4}
\begin{enumerate}
    \item $\cbrak{\pm i}$ 
    \item $\cbrak{\pm i,\pm 1}$
    \item $\cbrak{\pm 1}$
    \item $\cbrak{0,\pm i}$
\end{enumerate}
\end{multicols}
\item The coefficient of $ \brak{\text{z} - \pi}^{2}$ in the Taylor series expansion of
$
		f(\text{z}) = \begin{cases} \frac{\sin \text{z}}{\text{z} - \pi} & \text{if}\,\,\text{z} \neq \pi \\ -1 & \text{if}\,\,\text{z} = \pi \end{cases}
$
around $\pi$ is
\begin{multicols}{4}
\begin{enumerate}
    \item $\frac{1}{2}$ 
    \item $ -\frac{1}{2} $
    \item $ \frac{1}{6}$ 
    \item $ -\frac{1}{6}$ 
\end{enumerate}
\end{multicols}
\item  Consider $ \mathbb{R}^{2} $ with the usual topology. Which of the following statements are TRUE for all $ A, B \subseteq \mathbb{R}^{2}$?
\begin{itemize}
    \item[P:]$ A \cup B = \overline{A} \cup \overline{B} $.
    \item[Q:]$ A \cap B = \overline{A} \cap \overline{B} $.
    \item[R:]$ (A \cup B)^o = A^o \cup B^o $.
    \item[S:]$ (A \cap B)^o = A^o \cap B^o $.
\end{itemize}
\begin{multicols}{4}
\begin{enumerate}
    \item P and R only
    \item P and S only
    \item Q and R only
    \item Q and S only
\end{enumerate}
\end{multicols}
\item Let $ f : \mathbb{R} \to \mathbb{R} $ be a continuous function with $ f{\brak1} = 5 $ and $ f\brak{3} = 11 $. If $ g\brak{x} = \int_1^3 f\brak{x + t} \, dt $ then $ g^{\prime}\brak{0} $ is equal to \underline{\hspace{2cm}}.
\\
\item  Let  $P$  be a $ 2 \times 2 $ complex matrix such that $ trace\brak{P} = 1 $ and $ \det(P) = -6 $. Then, $ trace\brak{P^{4} - P^{3}} $ is \underline{\hspace{2cm}}.
\\
\item  Suppose that  $R$  is a unique factorization domain and that $ a, b \in R $ are distinct irreducible elements. Which of the following statements is \textbf{TRUE}?
\begin{enumerate}
    \item The ideal \( \langle 1 + a \rangle \) is a prime ideal.
    \item The ideal \( \langle a + b \rangle \) is a prime ideal.
    \item The ideal \( \langle 1 + ab \rangle \) is a prime ideal.
    \item The ideal \( \langle a \rangle \) is not necessarily a maximal ideal.
\end{enumerate}

\item  Let  $X$  be a compact Hausdorff topological space and let  $Y$  be a topological space. Let $ f : X \to Y $ be a bijective continuous mapping. Which of the following is \textbf{TRUE}?
\begin{enumerate}
    \item $f$ is a closed map but not necessarily an open map.
    \item $f$ is an open map but not necessarily a closed map.
    \item $f$ is both an open map and a closed map.
    \item $f$ need not be an open map or a closed map.
\end{enumerate}

\item Consider the linear programming problem:
\[
Maximize  x + \frac{3}{2} y
\]
\[
 subject\,\,to\,\,2x + 3y \leq 16,
\]
\[
x + 4y \leq 18,
\]
\[
x \geq 0, \; y \geq 0.
\]
If $S$ denotes the set of all solutions of the above problem, then
\begin{multicols}{2}
\begin{enumerate}
    \item $S$ is empty.
    \item $S$ is a singleton.
    \item $S$ is a line segment.
    \item $S$ has positive area.
\end{enumerate}
\end{multicols}
\item Which of the following groups has a proper subgroup that is \textbf{NOT} cyclic?
\begin{enumerate}
    \item  $ \mathbb{Z}_{15} \times \mathbb{Z}_{77} $
    \item  $ S_3 $
    \item  $ \brak{\mathbb{Z}, +} $
    \item  $ \brak{\mathbb{Q}, +} $
\end{enumerate}

\item The value of the integral
\[
\int_0^{\infty} \int_x^{\infty}  \brak{\frac{1}{y}} e^{-y/2} \, dy \, dx
\]
is \underline{\hspace{2cm}}.
\\
\item Suppose the random variable $U$ has uniform distribution on $\sbrak{0,1}$ and $ X = -2 \log U $. The density of $X$ is
\begin{enumerate}
    \item $ f\brak{x} = \begin{cases} e^{-x} & if\, x > 0 \\ 0 & otherwise \end{cases} $
    \item $ f\brak{x} = \begin{cases} 2e^{-2x} & if/, x > 0 \\ 0 & otherwise \end{cases} $
    \item $ f\brak{x} = \begin{cases} \frac{1}{2} e^{-x/2} & if\,  x > 0 \\ 0 & otherwise \end{cases} $
    \item $ f\brak{x} = \begin{cases} \frac{1}{2} & if\, x \in [0, 2] \\ 0 & otherwise \end{cases}\\ $
\end{enumerate}
\item  Let  $f$  be an entire function on $ \mathbb{C} $ such that $ |f(z)| \leq 100 \log |z| $ for each $ z $ with $ |z| \geq 2 $. If $ f(i) = 2i $, then $ f(1) $
\begin{enumerate}
    \item must be $2$ 
    \item must be $2i$
    \item must be $i$ 
    \item cannot be determined from the given data
\end{enumerate}
\item The number of group homomorphisms from $ \mathbb{Z}_{3} $ to $ \mathbb{Z}_{9} $ is \underline{\hspace{2cm}}.
\end{enumerate}
\end{document}
